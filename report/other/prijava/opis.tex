Diplomska naloga je ponovna implementacija in nadgradnja interaktivne umetniške instalacije ``15 sekund slave''~\cite{15secLeonardo}. Motivacija za instalacijo je umetniško delovanje ameriškega pop-art umetnika Andyja Warhola. ``15 sekund slave'' izgleda kot klasična slika, a je dejansko računalniški zaslon, okvirjen kot umetniška slika. Nad zaslonom je v okviru vgrajen digitalni fotoaparat, ki je povezan z računalnikom v ozadju. Vsakih 15 sekund fotoaparat slika obiskovalce galerije, ki stojijo pred sliko. Na sliki računalniški program poišče vse obraze in nato naključno izbere enega izmed njih. Ta obraz nato z grafičnimi filtri program obdela tako, da pridobi tako imenovani ``pop-art'' videz z manjšim številom živih barv, ki spominjajo na slike slavnih osebnosti, ki jih je iz fotografij delal Andy Warhol. Ker je prvotna instalacija nastala pred več kot 10 leti in se je strojna oprema v tem času že zelo spremenila, se je pokazala potreba po prilagoditvi aplikacije novemu stanju tehnologije \cite{trifonova}.

V magistrski nalogi najprej preučite splošne probleme pri vzdrževanju umetniških instalacij, ki temeljijo na računalniški tehnologiji \cite{miller1,miller2,digitalartconservation}. Zaradi hitrega napredka računalniške tehnologije je potrebno take aplikacije po eni strani prilagajati novi strojni in sistemski programski opremi, pa tudi novim funkcionalnim možnostim, ki jih nove tehnologije nudijo. Po drugi strani, pa z umetniškega vidika običajno želimo, da se zunanja pojava umetniškega dela ne spremeni. V konkretnem primeru instalacije ``15 sekund slave'' namesto računalnika in digitalnega fotoaparata uporabite mobilni telefon z vgrajeno kamero ter možnostjo brezžičnega prenosa podatkov. V enem načinu delovanja nove implementacije instalacije, se naj zunanji izgled, način uporabe in generirane slike ne razlikujejo od prvotne instalacije. V drugem načinu delovanja nove implementacije pa poiščite nove, kreativne načine funkcioniranja, ki jih omogoča novejša tehnologija. Tu je mišljena predvsem uporaba videa namesto statične slike in povezovanje s socialnimi omrežji.