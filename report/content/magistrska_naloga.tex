\chapter{Uvod}
Diplomska naloga je ponovna implementacija in nadgradnja interaktivne umetniške instalacije ``15 sekund slave''~\cite{15secLeonardo}. Motivacija za instalacijo je umetniško delovanje ameriškega pop-art umetnika Andyja Warhola. ``15 sekund slave'' izgleda kot klasična slika, a je dejansko računalniški zaslon, okvirjen kot umetniška slika. Nad zaslonom je v okviru vgrajen digitalni fotoaparat, ki je povezan z računalnikom v ozadju. Vsakih 15 sekund fotoaparat slika obiskovalce galerije, ki stojijo pred sliko. Na sliki računalniški program poišče vse obraze in nato naključno izbere enega izmed njih. Ta obraz nato z grafičnimi filtri program obdela tako, da pridobi tako imenovani ``pop-art'' videz z manjšim številom živih barv, ki spominjajo na slike slavnih osebnosti, ki jih je iz fotografij delal Andy Warhol. Ker je prvotna instalacija nastala pred več kot 10 leti in se je strojna oprema v tem času že zelo spremenila, se je pokazala potreba po prilagoditvi aplikacije novemu stanju tehnologije \cite{trifonova}.

V magistrski nalogi najprej preučite splošne probleme pri vzdrževanju umetniških instalacij, ki temeljijo na računalniški tehnologiji \cite{miller1,miller2,digitalartconservation}. Zaradi hitrega napredka računalniške tehnologije je potrebno take aplikacije po eni strani prilagajati novi strojni in sistemski programski opremi, pa tudi novim funkcionalnim možnostim, ki jih nove tehnologije nudijo. Po drugi strani, pa z umetniškega vidika običajno želimo, da se zunanja pojava umetniškega dela ne spremeni. V konkretnem primeru instalacije ``15 sekund slave'' namesto računalnika in digitalnega fotoaparata uporabite mobilni telefon z vgrajeno kamero ter možnostjo brezžičnega prenosa podatkov. V enem načinu delovanja nove implementacije instalacije, se naj zunanji izgled, način uporabe in generirane slike ne razlikujejo od prvotne instalacije. V drugem načinu delovanja nove implementacije pa poiščite nove, kreativne načine funkcioniranja, ki jih omogoča novejša tehnologija. Tu je mišljena predvsem uporaba videa namesto statične slike in povezovanje s socialnimi omrežji.




\chapter{Obdelava slik}
Instalacija ``15 sekund slave'' vsebuje 17 različnih filtrov~\cite[Poglavje~5]{diplomskaSamoJuvan} za obdelavo prejetih slik v pop-art slike. Filtri so bili izdelani s pomočjo grafičnega programa GIMP.

\section{GIMP}
GIMP je odprto-kodno programsko orodje za obdelavo slik.

\subsection{Zgodovina}
Začetek projekta sega v leto 1995, kot semestrski projekt v Univerzi Kalifornije, Berkeley. Kot prva avtorja, Spencer Kimball in Peter Mattis sta projekt poimenovala \textit{angl. General Image Manipulation Program}. Malo kasneje, ko je univerzo obiskal Richard Stallman sta ga prosila, če lahko zamenjata besedo ``General'' v ``GNU''. Od takrat naprej se program imenuje \textit{angl. GNU Image Manipulation Program} ali na kratko še vedno GIMP~\cite{wiki:GIMP}.

\section{Izbira filtrov}
Že prvotna izbira filtrov je izhajala iz programskega orodja za predelavo slik GIMP. Zaradi lažjega dela se je najprej grafičnega vmesnika izbralo zaporedje različnih filtrov z različnimi parametri. Izbrani so bili naslednji filtri:
\begin{itemize}
	\item \textbf{uravnovešenje barv} \textit{angl. color balance} \hfill \\ Prekrije celotno sliko z barvo v izbranem odtenku.
	\item \textbf{posteriziranje} \textit{angl. posterize} \hfill \\ Zmanjšuje število barv na sliki.
	\item \textbf{uravnovešanje barv v HSL prostoru} \textit{angl. hue saturation balance} \hfill \\ Spreminja odtenke barv, svetlost in intenzivnost na sliki.%
\end{itemize}

\section{Implementacija filtrov}
Ker je GIMP odprto-koden smo imeli tudi dostop do implementacije filtrov. Implementacije filtrov so napisani v programskem jeziku C. Na prvi pogled kot nalašč za enostaven prenos na Android platformo preko JNI. Vendar pa so se pokazale težave, saj so bili filtri močno povezani z GIMP objekti ter nekoliko bolj kompleksni, kot smo mi potrebovali. Zato smo se odločili, da vzamemo samo algoritem oziroma idejo za tem, ter implementiramo po naših potrebah. Naredili smo tri različne implementacije in jih med seboj primerjali.

Prva implementacija je napisana v programskem jeziku java. Izkazalo se je kot zelo počasno. Orodje za sproščanje pomnilnika GC \textit{(angl. garbage collector)} se je klical velikokrat, kar je posledično porabilo več kot nekaj sekund za filter. Tega si nismo mogli privoščiti, zato smo iskali alternative. 

Naslednja implementacija je napisana s pomočjo grafične kartice in sicer z orodjem OPENGL ES, verzije 2.0. Čas porabe pri izvajanju se je občutno zmanjšal. Ko so bile teksture zapisane na grafični kartici so se filtri izvajali v manj kot sekundi. Ta rešitev je bila že dovolj dobra, vendar pa se nam je zdelo, da je uporaba grafične kartice za tako lahke operacije potrata energije.  

Čeprav so bili rezultati že dovolj dobri, smo se odločili še za implementacijo v programskem jeziku C. Rezultati so pokazali, da je ta rešitev malo bolj počasna kot prejšnja, vendar pa za to nismo potrebovali grafične kartice na telefonu. Čas porabe je bil manj kot sekundo za filter.

\subsection{Uravnovešenje barv}

Ker se GIMP uspešno še naprej razvija, se je posledično tudi implementacija teh filtrov rahlo spremenila. 

Zato smo se odločili, da tudi tukaj naredimo rahlo spremembo. 


\chapter{TODO}
Datoteka {\tt magistrska\_naloga.tex} na kratko opisuje, kako se pisanja magistrskega dela lotimo z uporabo programskega pateka \LaTeX. V tem dokumentu bomo predstavili nekaj njegovih prednosti in hib. Kar se slednjih tiče, mi pride na misel ena sama. Ko se srečamo z njim, nam izgleda kot kislo jabolko, nismo prepričani, da bi želeli vanj ugrizniti. Lahko pa z njim pripravimo odličen zavitek ali pa pridemo na okus.

Česa od tega dokumenta ne pričakujte? Izkušeni uporabniki \LaTeX{}a bi vse skupaj zastavili
drugače. Morda bi napisali posebno razredno datoteko (\emph{class file}) --- v resnici priredili katero od obstoječih ---, v datoteki {\tt magistrska\_naloga.tex} ohranili samo najbolj grobo strukturo in vanjo vključevali  posamezna po\-glav\-ja. Hkrati s pisanjem teksta bi poskrbeli tudi za stvarno kazalo ({\tt makeindex}), literaturo pa bi citirali z uporabo {\BibTeX}{a}. Tega, skratka, v tem dokumentu ne boste našli.

Kaj vseeno najdemo. V Poglavju~\ref{ch1} bomo na hitro spoznali besedilne konstrukte kot so izreki, enačbe in dokazi. Naučili se bomo, kako se na njih sklicujemo. Poglavje~\ref{ch2} bo predstavilo vključevanje plovk: slik in tabel. V Poglavju~\ref{ch3} se bomo srečali s sklicevanjem na literaturo.
Sledil bo samo še zaključek.

\chapter{Sklicevanje na besedilne konstrukte}
\label{ch1}
Matematična ali popolna indukcija je eno prvih orodij, ki jih spoznamo za dokazovanje trditev pri matematičnih predmetih.
\begin{izrek}
\label{iz:1}
Za vsako naravno število $n$ velja
\begin{equation}
n < 2^n.
\label{eq:1}
\end{equation}
\end{izrek}
\begin{dokaz}
Dokazovanje z indukcijo zahteva, da neenakost~\eqref{eq:1} najprej preverimo za najmanjše naravno število --- $0$. Res, ker je $0 < 1 = 2^0$, je neenačba~\eqref{eq:1} za $n=0$ izpolnjena.

Sledi indukcijski korak. S predpostavko, da je neenakost~\eqref{eq:1} veljavna pri nekem naravnem številu $n$, je potrebno pokazati, da je ista neenakost v veljavi tudi pri njegovem nasledniku --- naravnem številu $n+1$. Računajmo.
\begin{align}
n+1 &< 2^n + 1  \label{eq:2}\\
    &\le 2^n + 2^n \label{eq:3}\\
    &= 2^{n+1} \nonumber
\end{align}
Neenakost~\eqref{eq:2} je posledica indukcijske predpostavke, neenakost~\eqref{eq:3} pa enostavno dejstvo, da je za vsako naravno število $n$ izraz $2^n$ vsaj tako velik kot 1. S tem je dokaz Izreka~\ref{iz:1} zaključen.
\end{dokaz}

Opazimo, da je \LaTeX\ številko izreka podredil številki poglavja.


\chapter{Plovke: slike in tabele}
\label{ch2}
Slike in daljše tabele praviloma vključujemo v dokument kot plovke. Pozicija plovke v končnem izdelku ni pogojena s tekom besedila, temveč z izgledom strani. \LaTeX\ bo skušal plovko postaviti samostojno, praviloma na vrh strani, na kateri se na takšno plovko prvič sklicujemo. Pri tem pa bo na vsako stran končnega izdelka želel postaviti tudi sorazmerno velik del besedila. V skrajnem primeru, če imamo res preveč plovk, se bo odločil za stran popolnoma zapolnjeno s plovkami.

\section{Formati slik}
Bitne slike, vektorske slike, kakršnekoli slike, z \LaTeX{}om lahko vključimo vse.
Slika~\ref{pic1} je v {\tt .pdf} formatu.
\begin{figure}
    \begin{center}
        \includegraphics[width=10cm]{pic1.pdf}
    \end{center}
\caption{Herschelov graf, vektorska grafika.}
\label{pic1}
\end{figure}
Pa res lahko vključimo slike katerihkoli formatov? Žal ne. Programski paket \LaTeX\ lahko uporabljamo v več dialektih. Ukaz {\tt latex} ne mara vključenih slik v formatu Portable Document Format {\tt .pdf}, ukaz {\tt pdflatex} pa ne prebavi slik v Encapsulated Postscript Formatu {\tt .eps}.
Strnjeno v Tabeli~\ref{tbl:1}.

\begin{table}
    \begin{center}
        \begin{tabular}{l|ccc}
            ukaz/format & {\tt .pdf} & {\tt .eps} & ostali formati \\ \hline
                        {\tt pdflatex} & da & ne & da \\
                        {\tt latex}   & ne & da  & da
        \end{tabular}
    \end{center}
\caption{}
\label{tbl:1}
\end{table}

Nasvet? Odločite se za uporabo ukaza {\tt pdflatex}. Vaš izdelek bo brez vmesnih stopenj na voljo v {.pdf} formatu in ga lahko odnesete v vsako tiskarno. Če morate na vsak način vključiti sliko, ki jo imate v {\tt .eps} formatu, jo vnaprej pretvorite v alternativni format, denimo {\tt .pdf}.

Včasih se da v okolju za uporabo programskega paketa \LaTeX\ nastaviti na kakšen način bomo prebavljali vhodne dokumente. Spustni meni na Sliki~\ref{pic2} odkriva uporabo \LaTeX{}a v njegovi pdf inkarnaciji --- {\tt pdflatex}.
\begin{figure}
\begin{center}
\includegraphics[width=10cm]{pic2.png}
\end{center}
\caption{Kateri dialekt uporabljati?}
\label{pic2}
\end{figure}

Vključena Slika~\ref{pic2} je seveda bitna.

Kaj pa stran iz študentskega referata?\label{pp}
Tudi njo lahko vključimo v dokument. Toda ne kot plovko.


%\chapter{}

\chapter{Kaj pa literatura?}
\label{ch3}
Kot smo omenili že v uvodu, je pravi način za citiranje literature uporaba \BibTeX{}a~\cite{bib}.
Programski paket \LaTeX je prvotno predstavljen v priročniku~\cite{lat} in je v resnici nadgradnja sistema \TeX\ avtorja Donalda Knutha, znanega po denimo, če izpustim njegovo umetnost programiranja, Knuth-Bendixovem algoritmu~\cite{dk1}.

Vsem raziskovalcem s področja računalništva pa svetujem v branje mnenje L.\ Fortnowa~\cite{lf}~\cite{trifonova}.

\chapter{Sklepne ugotovitve}
Izbira \LaTeX\ ali ne \LaTeX\ je seveda prepuščena vam samim. Res je, da so prvi koraki v \LaTeX{}u težavni. Ta dokument naj vam služi kot začetna opora pri hoji.
